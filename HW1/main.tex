\documentclass[12pt,onecolumn]{article}
\usepackage[utf8]{inputenc} % UTF8 input encoding
\usepackage[T2A]{fontenc}   % T2A font encoding for Cyrillic script
\usepackage[russian]{babel} % Russian language support
\usepackage{listings}
\usepackage{float}
\usepackage{mathtools}
\everymath{\displaystyle}
\usepackage{listings} 
\usepackage[usenames]{color}
\usepackage{hyperref}
\usepackage{geometry}
\usepackage{verbatim}
\newcommand{\nparagraph}[1]{\paragraph{#1}\mbox{}\\}
\geometry{
  a4paper,
  top=20mm, 
  right=20mm, 
  bottom=20mm, 
  left=25mm
}

\begin{document}
\setcounter{tocdepth}{4}
\begin{center}
    Федеральное государственное автономное образовательное учреждение высшего образования "Национальный Исследовательский Университет ИТМО"\\ 
    Мегафакультет Компьютерных Технологий и Управления\\
    Факультет Программной Инженерии и Компьютерной Техники \\
    \includegraphics[scale=0.3]{image/itmo.jpg} % нужно закинуть картинку логтипа в папку с отчетом
\end{center}
\vspace{1cm}


\begin{center}
    \large \textbf{Вариант №5}\\
    \textbf{Домашняя работа 1}\\
    по дисциплине\\
    \textbf{'Разработка компиляторов'}
\end{center}

\vspace{2cm}

\begin{flushright}
  Выполнил Студент  группы P33102\\
  \textbf{Лапин Алексей Александрович}\\
  Преподаватель: \\
  \textbf{Лаздин Артур Вячеславович}\\
\end{flushright}

\vspace{9cm}
\begin{center}
    г. Санкт-Петербург\\
    2024г.
\end{center}
\pagestyle{empty}

\newpage
\tableofcontents
\newpage

\section*{Задание}
Для каждой грамматики из списка, соответствующему варианту, выполнить действия,
указанные в таблице заданий, определить тип грамматики по классификации Хомского, и
для грамматик типа 2 и 3 постройте вывод не менее двух предложений.
Укажите язык, порождаемый грамматикой, в множественно-теоретическом виде.
\nparagraph{№6}
\textbf{Грамматика:}\\
$S \to aB ~|~ \varepsilon$\\
$B \to bS ~|~ bA$\\
$A \to aA ~|~ \varepsilon$\\
\textbf{Тип грамматики: } Контекстно-свободная
\begin{enumerate}
    \item $S \to \varepsilon$
    \item $S \to aB \to abA \to ab$
    \item $S \to aB \to abS \to abaB \to ababA \to abab$
    \item $S \to aB \to abS \to abaB \to ababA \to ababaA \to ababaaA \to ababaaa$
\end{enumerate}
\textbf{Язык:} $L(G) = \{\left(ab\right)^n a^m, n \geq 0, m \geq 0\}$
\nparagraph{№12}
\textbf{Грамматика:}\\
$S \to abC~ |~ aB$ \\
$B \to bc$\\
$bC \to bc$\\
\textbf{Тип грамматики: } Контекстно-зависимая
\begin{enumerate}
    \item $S \to aB \to abc$
    \item $S \to abC \to abc$
\end{enumerate}
\textbf{Язык:} $L(G) = {abc}$
\nparagraph{№16}
$S \to aS~ |~ bB~ |~ \varepsilon$\\
$B \to aB~ |~ bS~ |~ bC$\\
$C \to aC~ |~ \varepsilon$\\
\textbf{Тип грамматики: } Контекстно-свободная
\begin{enumerate}
    \item $S \to \varepsilon$
    \item $S \to bB \to bbC \to bb$
    \item $S \to bB \to bbC \to bbaC \to bbaaC \to bbaaaC \to bbaaa$
    \item $S \to bB \to baB \to baaB \to baaaB \to baaaaB \to baaaabC \to baaaabaC \to baaaabaaC \to baaaabaa$
    \item $S \to aS \to aaS \to aaaS \to aaa$
    \item $S \to aS \to aaS \to aabB \to aabbS \to aabbaS \to aabbaS \to aabbaaS \to aabbaa$
    \item $S \to aS \to aaS \to aabB \to aabbS \to aabbaS \to aabbaaS \to aabbaabB\to aabbaabaB \to aabbaabaaB \to aabbaabaabC \to aabbaabaab$
    \item $S \to aS \to aaS \to aabB \to aabbS \to aabbaS \to aabbaS \to aabbaaS \to aabbaabB \to aabbaabbS \to aabbaabbbB \to aabbaabbbbC \to aabbaabbbb$
    \item $S \to aS \to aaS \to aabB \to aabbS \to aabbaS \to aabbaS \to aabbaaS \to aabbaabB \to aabbaabbS \to aabbaabbbB \to aabbaabbbbC \to aabbaabbbbaC \to aabbaabbbbaaC \to aabbaabbbbaaaC \to aabbaabbbbaaa$
\end{enumerate}
\textbf{Язык:} Символы $b$ или $a$ в любом порядке, количество $b$ четно. 
\nparagraph{№23}
$S \to bSS | ab$\\
\textbf{Тип грамматики: } Контекстно-свободная
\begin{enumerate}
    \item $S \to ab$
    \item $S \to bSS \to bSab \to babab$
    \item $S \to bSS \to bSab \to bbSSab \to bbabSab \to bbabbSSab \to bbabbSabab \to bbabbababab$
\end{enumerate}
\textbf{Язык: } Последовательности символов $b$ и $ab$, количество $ab$ четно.
\nparagraph{№23} 
$S \to SS~ |~ RS$\\
$R \to RR~ |~ 0$\\
$RS \to SR$\\
$0S0 \to 010$\\
\textbf{Тип грамматики: } Контекстно-зависимая\\
\begin{enumerate}
    \item $S \to RS \to 0S \to 0SS\to 0RSS\to 0SRS \to 0S0S \to 010S \to 010RS \to 010SR \to 010S0 \to 01010$
    \item $S \to RS \to 0S \to 0RS\to 0SR \to 0S0 \to 010$
\end{enumerate}
\textbf{Язык: }  $L(G) = \{0(10)^n, n > 0\}$

\end{document}